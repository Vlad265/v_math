\begin{table}[H]
    \renewcommand{\tablename}{Таблица}
    \caption{Сравнение методов по точности и количеству итераций для матрицы Finan512}
    \label{tab:table5}
    \begin{tabularx}{1\textwidth}{
        | >{\centering\arraybackslash}X
        | >{\centering\arraybackslash}X
        | >{\centering\arraybackslash}X
        | >{\centering\arraybackslash}X
        | >{\centering\arraybackslash}X
        | >{\centering\arraybackslash}X
        | >{\centering\arraybackslash}X |
    }
        \hline
        \multirow{Название метода} &
        \multicolumn{2}{X|}{Без предобуславливателя} &
        \multicolumn{2}{X|}{С предобуславливателем неполное разложение Холецкого} &
        \multicolumn{2}{X|}{С предобуславливателем LU-разложение} \\
        \cline{2-7}
        & Число итераций & Точность & Число итераций & Точность & Число итераций & Точность \\
        \hline
        bicg        &  70 & 10^{-8} & 9 & 10^{-8} & 9 & 10^{-8}  \\
        \hline
        bicgstab    & 100 & 10^{-8} & 8 & 10^{-8} & 9 & 10^{-8} \\
        \hline
        bicgstabl   & 95 & 10^{-8} & 8 & 10^{-8} & 9 & 10^{-8} \\
        \hline
        cgs         & 36 & 10^{-8} & 8 & 10^{-8} & 4 & 10^{-8} \\
        \hline
        gmres       & 65 & 10^{-8} & 8 & 10^{-8} & 10 & 10^{-5} \\
        \hline
        lsqr        & 500 & 10^{-8} & 8 & 10^{-8} & 9 & 10^{-8} \\
        \hline
        minres      & 65 & 10^{-8} & 8 & 10^{-8} & 8 & 10^{-8} \\
        \hline
        pcg         & 67 & 10^{-8} & 8 & 10^{-8} & 8 & 10^{-8} \\
        \hline
        qmr         & 64 & 10^{-8} & 8 & 10^{-8} & 8 & 10^{-8} \\
        \hline
        symmlq      & 67 & 10^{-6} & 7 & 10^{-8} & 8 & 10^{-7} \\
        \hline
        tfqmr       & 75 & 10^{-8} & 7 & 10^{-8} & 8 & 10^{-8} \\
        \hline
    \end{tabularx}
\end{table}
