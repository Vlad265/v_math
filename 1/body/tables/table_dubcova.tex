\begin{table}[H]
    \renewcommand{\tablename}{Таблица}
    \caption{сравнение методов по точности и количеству итерацй для матрицы Dubcova2}
    \label{tab:table1}
    \begin{tabularx}{1\textwidth}{
        | >{\centering\arraybackslash}X
        | >{\centering\arraybackslash}X
        | >{\centering\arraybackslash}X
        | >{\centering\arraybackslash}X
        | >{\centering\arraybackslash}X
        | >{\centering\arraybackslash}X
        | >{\centering\arraybackslash}X |
    }
        \hline
        \multirow{Название метода} &
        \multicolumn{2}{X|}{Без предобуславливателя} &
        \multicolumn{2}{X|}{С предобуславливателем неполное разложение Холецкого} &
        \multicolumn{2}{X|}{С предобуславливателем LU-разложение} \\
        \cline{2-7}
        & Число итераций & Точность & Число итераций & Точность & Число итераций & Точность \\
        \hline
        bicg & 180 & 10^{-8} & 148 & 10^{-8} & 148 & 10^{-8}  \\
        \hline
        bicgstab & 250 & 10^{-8} & 148 & 10^{-8} & 204 & 10^{-8} \\
        \hline
        bicgstabl & 270 & 10^{-8} & 150 & 10^{-8} & 180 & 10^{-8} \\
        \hline
        cgs & 152 & 10^{-8} & 148 & 10^{-8} & 104 & 10^{-8} \\
        \hline
        gmres & 180 & 10^{-8} & 148 & 10^{-8} & 147 & 10^{-8} \\
        \hline
        lsqr & 2000 & 10^{-8} & 144 & 10^{-8} & 2300 & 10^{-8} \\
        \hline
        minres & 180 & 10^{-8} & 147 & 10^{-8} & 140 & 10^{-8} \\
        \hline
        pcg & 180 & 10^{-8} & 147 & 10^{-8} & 147 & 10^{-8} \\
        \hline
        qmr & 178 & 10^{-8} & 145 & 10^{-8} & 147 & 10^{-8} \\
        \hline
        symmlq & 180 & 10^{-5} & 149 & 10^{-6} & 148 & 10^{-6} \\
        \hline
        tfqmr & 290 & 10^{-8} & 150 & 10^{-8} & 216 & 10^{-8} \\
        \hline
    \end{tabularx}
\end{table}
