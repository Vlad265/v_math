\section*{Выводы}

В ходе выполнения лабораторной работы были решены системы линейных уравнений,
где левой частью были матрицы различной размерности, а правой единичные векторы,
различными способами.
Решение систем линейных уравнений большой размерности без предобуславливателя выполняется
за большее количество итераций по сравнению с предобуславливателем неполного разложения
Холецкого или с предобуславливателем LU-разложения.
За меньшее количество итераций систему на основе матрицы Dubcova2 решает
методом lsqr с предобуславливателем неполное разложение Холецкого (145 итераций).
На основе матрицы Finan512 - cgs с предобуславливателем LU-разложение (5 итераций).
G2\_circuit – bicgstabl с предобуславливателем неполное разложение Холецкого.
Qa8fm - среднем всем методам с предобуславливателем понадобилось
примерно одинаковое количество итераций (в среднем 8).
Thermomech\_dM - cgs с предобуславливателем LU-разложение (6 итераций).